% for ensemble averages
% TODO: Convert to \DeclarePairedDelimiter?
\newcommand{\avg}[1]{\left\langle #1 \right\rangle}
\newcommand{\savg}[1]{\langle #1 \rangle}
% https://tex.stackexchange.com/q/448/9945
\newcommand{\cavg}[2]{\left\langle #1 \ \middle|\ #2 \right\rangle}
% https://tex.stackexchange.com/q/187162/9945
\newcommand{\scavg}[2]{\langle #1 \mid #2 \rangle}

% for thermodynamic partial derivatives
\newcommand{\dpdc}[3]{\left(\dpd{#1}{#2}\right)_{#3}}

% for material derivatives
\newcommand{\D}[2]{\frac{\Dif\kern1pt#1}{\Dif\kern1pt#2}}

% inline derivatives
\newcommand{\ipd}[2]{\partial\kern1pt#1 / \partial\kern1pt#2}
\newcommand{\iod}[2]{\dif\kern1pt#1 / \dif\kern1pt#2}
\newcommand{\iosecd}[2]{\dif^{\kern2pt 2}\kern-2pt#1 / \dif\kern1pt#2^2}

% make \e be the same as \cdot 10^
\providecommand{\e}[1]{\ensuremath{\cdot 10^{#1}}}

% new math operators
\DeclareMathOperator{\sgn}{sgn}
\newcommand{\vectornorm}[1]{\left|\left|\mathrmbf{#1}\right|\right|}
\DeclareMathOperator{\atan}{atan}
\DeclareMathOperator{\erf}{erf}
\DeclareMathOperator{\erfinv}{erfinv}
\DeclareMathOperator{\LambertW}{W}
\newcommand{\mytilde}{~\raise.17ex\hbox{$\scriptstyle\mathtt{\sim}$}~}
\newcommand{\Order}[1]{\ensuremath{\mathcal{O}\left(#1\right)}}
\newcommand{\Ordersb}[1]{\ensuremath{\mathcal{O}\left[#1\right]}}
\newcommand{\sOrder}[1]{\ensuremath{\mathcal{O}(#1)}}
%\DeclarePairedDelimiter\abs{\lvert}{\rvert}% https://tex.stackexchange.com/a/43009/9945
%\newcommand{\abs}[1]{\lvert#1\rvert}

\newcommand{\Bo}{\mathrm{Bo}}
\newcommand{\Ec}{\mathrm{Ec}}
\newcommand{\Fr}{\mathrm{Fr}}
\newcommand{\Oh}{\mathrm{Oh}}
\newcommand{\Pe}{\mathrm{Pe}}
\renewcommand{\Pr}{\mathrm{Pr}}
\newcommand{\Real}{\Re~}
\renewcommand{\Re}{\mathrm{Re}}
\newcommand{\Tu}{\mathrm{Tu}}
\newcommand{\We}{\mathrm{We}}
\newcommand{\Ma}{\mathrm{Ma}}

\newcommand{\Tubar}{\overline{\Tu}}
\newcommand{\Lambdabar}{\overline{\Lambda}}
\newcommand{\Ubar}{\overline{U}}
\newcommand{\kbar}{\overline{k}}
\newcommand{\epsbar}{\overline{\varepsilon}}
\newcommand{\ebar}{\overline{e}} % used for turbulent initial disturbance level theory in CDRSV review section

%\newcommand{\Unet}{\mathcal{U}}
\newcommand{\Unet}{U_\text{net}}

\newcommand{\Relo}{\Re_{\ell0}}
\newcommand{\Welo}{\We_{\ell0}}
\newcommand{\Wego}{\We_{\text{g}0}}
\newcommand{\Ohlo}{\Oh_{\ell0}}
\newcommand{\Welocrit}{\We_{\ell0,\text{crit}}}
\newcommand{\Wegocrit}{\We_{\text{g}0,\text{crit}}}
\newcommand{\Relocrit}{\Re_{\ell0,\text{crit}}}
\newcommand{\Relotrans}{\Re_{\ell0,\text{trans}}}
\newcommand{\Reloturb}{\Re_{\ell0,\text{turb}}}
\newcommand{\rhol}{\rho_{\ell}}
\newcommand{\rhog}{\rho_\text{g}}
\newcommand{\nul}{\nu_{\ell}}
\newcommand{\nug}{\nu_\text{g}}
\newcommand{\mul}{\mu_{\ell}}
\newcommand{\mug}{\mu_\text{g}}
\newcommand{\Prl}{\Pr_\ell}
% also need to change trajectory and jet breakup scripts when changing these
% 2018-11-30: search for _\text{l

\newcommand{\xbavg}{\savg{x_\text{b}}}
\newcommand{\tbavg}{\savg{t_\text{b}}}
\newcommand{\sigmaxb}{\sigma_{x_\text{b}}}
\newcommand{\xiavg}{\savg{x_\text{i}}}
\newcommand{\thetai}{{\theta_\text{i}}}
\newcommand{\udavg}{\savg{u_\text{d}}}
\newcommand{\vdavg}{\savg{v_\text{d}}}
\newcommand{\Dmax}{D_\text{max}}
\newcommand{\Frh}{\Fr_{h_0}}
\newcommand{\Xb}{X_\text{b}}
\newcommand{\Yb}{Y_\text{b}}
\newcommand{\Ub}{U_\text{b}}
\newcommand{\Ubvec}{\vec{U}_\text{b}}
\newcommand{\Vb}{V_\text{b}}
\newcommand{\Ud}{U_\text{d}}
\newcommand{\Udvec}{\vec{U}_\text{d}}
\newcommand{\Vd}{V_\text{d}}
\newcommand{\Xd}{X_\text{d}}
\newcommand{\Xdvec}{\vec{X}_\text{d}}
\newcommand{\Yd}{Y_\text{d}}
\newcommand{\Ugvec}{\vec{U}_\text{g}}
\newcommand{\Vg}{V_\text{g}}
\newcommand{\Uj}{U_\text{j}}
\newcommand{\Vj}{V_\text{j}}
\newcommand{\Xj}{X_\text{j}}
\newcommand{\Yj}{Y_\text{j}}
\newcommand{\Ad}{A_\text{d}}
\newcommand{\mdrop}{m_\text{d}}
\newcommand{\Cd}{C_\text{d}}
\newcommand{\taumax}{\tau_\text{max}}
\newcommand{\Relxtrans}{\Re_{\ell{}x,\text{trans}}}
\newcommand{\xtrans}{x_\text{trans}}
\newcommand{\Frr}{\Fr_\text{r}}
\newcommand{\dr}{d_\text{r}}
\newcommand{\utau}{u_\tau}
\newcommand{\utauo}{u_{\tau,0}}
% TODO: tan theta_i/2
% TODO: x_\text{b}
% TODO: x_\text{i}
% TODO: \overline{\varepsilon}

\newcommand{\Range}{R}
\newcommand{\height}{h}
\newcommand{\etaR}{\eta_R}
\newcommand{\etah}{\eta_h}
\newcommand{\Rs}{R^*}
\newcommand{\hs}{h^*}

\newcommand{\CLR}{C_\text{LR}}
\newcommand{\CTR}{C_\text{TR}}

\newcommand{\uprime}{{u^\prime}}
\newcommand{\vprime}{{v^\prime}}
\newcommand{\wprime}{{w^\prime}}
\newcommand{\uprimesub}[1]{u^\prime_{#1}}
\newcommand{\vprimesub}[1]{v^\prime_{#1}}
\newcommand{\uprimebar}[1]{\overline{\uprimesub{#1}}}
\newcommand{\uprimebarno}[1]{\overline{\uprime}}
\newcommand{\vprimebar}[1]{\overline{\vprimesub{#1}}}
\newcommand{\vprimebarno}{\overline{\vprime}}
\newcommand{\Tuexp}[1]{\Tu_0^{#1}}
\newcommand{\Tubarexp}[1]{\Tubar_0^{\kern1.0pt #1}}
\newcommand{\mdotp}{\dot{m}^{\prime}}
\newcommand{\mdotpp}{\dot{m}^{\prime\prime}}
\newcommand{\avgmdotpp}{\savg{\mdotpp}}

% https://tex.stackexchange.com/a/69017
\makeatletter
\DeclareRobustCommand{\Vol}{\text{\volumedash}V}
\newcommand{\volumedash}{%
  \makebox[0pt][l]{%
    \ooalign{\hfil\hphantom{$\m@th V$}\hfil\cr\kern0.08em--\hfil\cr}%
  }%
}
\makeatother
%\newcommand{\Vol}{\rotatebox[origin=c]{180}{\ensuremath{A}}}
\newcommand{\Volsub}{\Vol}
\newcommand{\Volsubsub}{\Vol}

% very large manually set bracket size
% https://tex.stackexchange.com/a/6796
\makeatletter
\newcommand{\vast}{\bBigg@{3}}
\newcommand{\Vast}{\bBigg@{4}}
\makeatother

% stacked text for underbraces
\newcommand{\ubtext}[2]{\stackrel{\text{\scriptsize#1}}{\text{\scriptsize#2}}}

% approximately proportional to symbol
% from https://tex.stackexchange.com/questions/33538/how-to-get-an-approximately-proportional-to-symbol
\newcommand{\appropto}{\mathrel{\vcenter{
  \offinterlineskip\halign{\hfil$##$\cr
    \propto\cr\noalign{\kern2pt}\sim\cr\noalign{\kern-2pt}}}}}

\newcommand*{\cancelzero}[1]{\cancelto{\scriptstyle 0}{#1}}

%% closed square, cube, etc., roots
%% https://tex.stackexchange.com/questions/29834/closed-square-root-symbo
%\let\oldsqrt\sqrt
%% it defines the new \sqrt in terms of the old one
%\def\sqrt{\mathpalette\DHLhksqrt}
%\def\DHLhksqrt#1#2{%
%\setbox0=\hbox{$#1\oldsqrt{#2\,}$}\dimen0=\ht0
%\advance\dimen0-0.2\ht0
%\setbox2=\hbox{\vrule height\ht0 depth -\dimen0}%
%{\box0\lower0.4pt\box2}}

% These are environments for equation, equation*, align, and align* that have not yet been checked by DOUBTS. These will return a warning unless changed to their corresponding real environment.
\newenvironment{uncheckedequation}
{\begin{equation}}
{\end{equation}\PackageWarning{DOUBTS}{Check with DOUBTS.}}

\newenvironment{uncheckedequation*}
{\begin{equation*}}
{\end{equation}\PackageWarning{DOUBTS}{Check with DOUBTS.}}

\newenvironment{uncheckedalign}
{\begin{align}}
{\end{align}\PackageWarning{DOUBTS}{Check with DOUBTS.}}

\newenvironment{uncheckedalign*}
{\begin{align*}}
{\end{align*}\PackageWarning{DOUBTS}{Check with DOUBTS.}}

\newcommand*{\tripleprime}{\prime\prime\prime}

% https://tex.stackexchange.com/a/97246/9945
\DeclareMathOperator{\sech}{sech}
\DeclareMathOperator{\csch}{csch}
\DeclareMathOperator{\arcsec}{arcsec}
\DeclareMathOperator{\arccot}{arccot}
\DeclareMathOperator{\arccsc}{arccsc}
\DeclareMathOperator{\arccosh}{arccosh}
\DeclareMathOperator{\arcsinh}{arcsinh}
\DeclareMathOperator{\arctanh}{arctanh}
\DeclareMathOperator{\arcsech}{arcsech}
\DeclareMathOperator{\arccsch}{arccsch}
\DeclareMathOperator{\arccoth}{arccoth} 
